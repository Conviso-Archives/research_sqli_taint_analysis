\section{Introduction}\label{sec:int}
Data Input Validation flaws have high impact over computational systems. This class of vulnerability can be even more dangerous on a WEB system due to it exposure. According OWASP (\textit{a.k.a.} Open WEB Application Security Project), Data Input Validation flaws have been the most frequent vulnerability on WEB applications in the last $10$ years \cite{OWASP2010}. This category of vulnerabilities represented $73\%$ of all uncovered flaws during code-review projects executed by Conviso \footnote{Conviso Application Security - http://www.conviso.com.br/} along $2012$.

One of the most frequent and harmful \textit{Data Input Validation} based vulnerability is \textit{SQL Injection}. This specific type of vulnerability allows attackers compromise data privacy and sometimes the whole host system through database. According CVE (\textit{a.k.a.} Common Vulnerabilities and Exposures) $83$ high severity SQL Injection vulnerabilities were uncovered in the last three months. Other interesting information is that most of these vulnerabilities targeted PHP based systems. We can also find this information looking for our internal statistics: $46\%$ of all code-review projects performed along $2012$ were over PHP based systems.

Due to the above mentioned numbers and aiming optimise our white-box test service, we decided to seek for methods which automatically identify such kind of vulnerability by analysing PHP source code. Then, this research describes a method for identifying SQL Injection flaws by using Backward Taint Analysis. We applied the proposed technique to analyse code of real world systems and benchmark code collected from NIST (\textit{a.k.a.} National Institute of Standards and Technology).

Section \ref{sec:the} presents theoretical background and references required for understanding the proposed method. Then Section \ref{sec:cont} presents the proposed method. In line with that, Section \ref{sec:exp} describes experiments results and finally Section \ref{sec:con} presents conclusions and future research.
